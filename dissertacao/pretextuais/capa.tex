
%-----------------------------------------------%
% Para alterar o gênero dos comandos orientador
% e coorientador.
%-----------------------------------------------%
% \renewcommand{\orientadorname}{Orientadora:}
\renewcommand{\coorientadorname}{Coorientadora:}
%-----------------------------------------------%


%-----------------------------------------------%
% Informações de dados para CAPA e FOLHA DE ROSTO
%-----------------------------------------------%
\titulo{Modelo de Trabalho Acadêmico com \abnTeX}
\autor{Nome do Aluno}
\local{Itajaí - SC}
\data{setembro/2023}
\orientador{Prof. Orientador da Silva, Dr.}
\coorientador{Profa. Fulana da Silva, Dra.}
\instituicao{%
  Universidade do Vale do Itajaí -- UNIVALI
  \par
  Campus Itajaí
  \par
  Mestrado Acadêmico em Computação Aplicada}
\tipotrabalho{dissertação (Graduação)}
%-----------------------------------------------%


%-----------------------------------------------%
% O preambulo deve conter o tipo do trabalho, o objetivo, o nome da instituição e a área de concentração 

\preambulo{Projeto de dissertação apresentado como requisito parcial para a obtenção do título de Mestre em Computação Aplicada pela Universidade do Vale do Itajaí.}

%\textoaprovacao{Este trabalho foi julgado adequado para obtenção do título de Mestre em Computação Aplicada, pelo Universidade do Vale do Itajaí, e aprovado na sua forma final pela comissão avaliadora abaixo indicada.}
%-----------------------------------------------%

%-----------------------------------------------%
% Estilo de cabeçalho que só contém o número da 
% página e uma linha
%-----------------------------------------------%
\makepagestyle{cabecalholimpo}
\makeevenhead{cabecalholimpo}{\thepage}{}{} % páginas pares
\makeoddhead{cabecalholimpo}{}{}{\thepage} % páginas ímpares
% \makeheadrule{cabecalholimpo}{\textwidth}{\normalrulethickness} % linha
%-----------------------------------------------%

