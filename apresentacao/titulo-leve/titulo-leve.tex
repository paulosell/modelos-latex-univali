% 2017-03-10 - Emerson Ribeiro de Mello - mello@ifsc.edu.br
% 2023-26-09 - Paulo Fylippe Sell - paulo.sell@edu.univali.br
% \documentclass[handout,xcolor=pdftex,dvipsnames,table]{beamer}
\documentclass{beamer}

% usando tema UNIVALI-Leve
\usepackage{beamerthemeUNIVALI-LEVE}


\usepackage[utf8]{inputenc}
\usepackage[T1]{fontenc}
\usepackage[english,brazil]{babel}

% Novos tipos de colunas em tabelas que permita definir a largura
\newcolumntype{R}[1]{>{\RaggedLeft\arraybackslash}p{#1}}
\newcolumntype{C}[1]{>{\centering\arraybackslash}p{#1}}
\newcolumntype{L}[1]{>{\RaggedRight\arraybackslash}p{#1}}
\usepackage{graphicx}
\usepackage{xcolor}
\usepackage{colortbl}


% -------------------------------------------------%
% Configurações para o PDF que será gerado
% -------------------------------------------------%
\hypersetup{
    colorlinks,
    pdffitwindow=false,  % window fit to page when opened
    pdfstartview={FitH}, % fits the width of the page to the window
    linkcolor=black,  %%% cor do tableofcontents, \ref, \footnote, etc
	citecolor=black,  %%% cor do \cite
	urlcolor=black   %%% cor do \url e \href
}
% Use esse comando para fazer links para sites.
% Os links ficarão em azual e dentro dos símbolos < e > 
\newcommand{\MYhref}[3][blue]{\href{#2}{\color{#1}{<#3>}}}%
% -------------------------------------------------%


% -------------------------------------------------%
%              Título 
% -------------------------------------------------%
\title{Modelo de apresentação UNIVALI}
\subtitle{Tema: Título Leve}
\author{Paulo Fylippe Sell}
\date{26 de setembro de 2023}
\institute{Mestrado em computação aplicada\\
	Universidade do Vale do Itajaí\\
	campus Itajaí\\
\url{paulo.sell@edu.univali.br}
}
% -------------------------------------------------%


\begin{document}

\begin{frame}[t]
    \maketitle
\end{frame}

% Descomente as linhas abaixo se desejar colocar um sumário
%\begin{frame}{Sumário}
%	\tableofcontents
%\end{frame}

%------------------------------------------------------------------------------------%
%                         Inicio do documento
%------------------------------------------------------------------------------------%


\section{Listas}

\begin{frame}{Apenas começando}
    \begin{enumerate}
        % usando o comando espaço para aumentar a distância entre os itens
        \espaco{1.5em}
        \item Primeiro item
              \begin{itemize}
                  \item Primeiro item
                  \item Segundo item
              \end{itemize}
        \item Segundo item
        \item Terceiro item
              \begin{itemize}
                  \item Primeiro item
                  \item Segundo item
              \end{itemize}
    \end{enumerate}
\end{frame}

% usuando o slide com a opcao wide
\begin{frame}[wide]{Apenas começando}
    \begin{enumerate}
        \item Primeiro item
              \begin{itemize}
                  \item Primeiro item
                  \item Segundo item
              \end{itemize}
        \item Segundo item
        \item Terceiro item
              \begin{itemize}
                  \item Primeiro item
                  \item Segundo item
              \end{itemize}
    \end{enumerate}
\end{frame}

\subsection{Blocos}


\begin{frame}{Blocos}
    \begin{block}{Esse é um bloco}
        Isso é um teste
    \end{block}
    \begin{block}{}
        Bloco sem título
    \end{block}
    \begin{alertblock}{Alerta}
        Esse é um alerta
    \end{alertblock}
\end{frame}


\begin{frame}[fragile]{Código em C e Java}

    \begin{itemize}
        \item Comandos criados para as seguintes linguagens
              \begin{itemize}
                  \item \texttt{ansic, java, shell, php, matlab, python, xml, sql}
                  \item \texttt{ansicp, javap, shellp, phpp, matlabp, pythonp, xmlp, sqlp}
                  \item Letra p no final indica que a fonte será \texttt{scriptsize}
              \end{itemize}
    \end{itemize}

    % incluindo o código de um arquivo externo
    \includecode{ansic}{codigos/ola.c}


    % escrevendo o código diretamente dentro do frame
    \javap
    \begin{lstlisting}
public static voi main(String args[]){
	System.out.println("Ola mundo");
}
\end{lstlisting}
\end{frame}

\begin{frame}[t]
    \maketitle
\end{frame}



\end{document}
